\documentclass[spanish]{udpreport}
\usepackage[utf8]{inputenc}
\usepackage[spanish]{babel}
\graphicspath{{images/}}
\usepackage{graphicx}
\usepackage{multirow}
\usepackage{verbatim} 
\usepackage{upgreek} 

% Podemos establecer el logo de alguna entidad o dejar el de la UDP (defecto)
\setlogo{EITFI}

\title{Metodologia RUP}
\author{Christian López \\ Thomas Muñoz \\ Flavio Pallini}
\email{christian.lopeza@mail.udp.cl \\ thomas.munoz@mail.udp.cl }
\date{29 de agosto de 2017}

% Además podemos establecer la facultad y escuela
% los valores por defecto son los siguientes:
\udpschool{Escuela de Informática y Telecomunicaciones}
\udpfaculty{Facultad de Ingeniería y Ciencias}
\udpuniversity{Universidad Diego Portales}

\begin{document}
\maketitle

\tableofcontents
\listoffigures

\chapter{Introducción}
Hoy en día es casi imposible realizar un proyecto de software sin hacer uso de alguna metodología, ya que estos requieren de un manejo de muchas variables, métodos y procesos. Es por esto que para satisfacer estas necesidades se hace uso de la ingeniería de software como pauta a seguir cuando se quiere modelar la solución a un problema de software en una empresa.\par
La metodología RUP, es un proceso de desarrollo de software creado por \textit{Rational Software}, y actualmente propiedad de IBM, obteniendo el nombre de \textit{Rational Unified Process}.\par
Es uno de los métodos de desarrollo de software más utilizados para el análisis, implementación y documentación de sistemas orientados a objetos. Se caracteriza por no ser estático, ya que está compuesto por un conjunto de fases adaptables según la necesidad del cliente o avance del equipo de desarrollo.

\chapter{Explicación de Metodología}
Esta metodología permite dividir el proceso de desarrollo en cuatro grandes fases donde cada una incluye modelamiento del negocio, análisis, diseño, construcción, pruebas e implantación.

\section{¿Por qué utilizar RUP?}
\label{sec: Por que utilizar RUP}
RUP proporciona información sobre lo que puede esperarse de la tarea de desarrollo. Ofrece un glosario de terminología y una enciclopedia de conocimiento que le ayuda a comunicar sus necesidades de forma eficaz al equipo de  desarrollo de software. \par
Para un gestor o jefe de equipo, RUP proporciona  un proceso el cual le permite comunicarse de forma eficaz con el personal, gestionar la planificación y el control de su trabajo. Para un Diseñador, RUP proporciona una buena base de arquitectura y una gran cantidad de materiales con las que construir una definición de un proceso, lo que le permite configurar y ampliar dicha base como desee.

\section{¿Cuándo debo utilizar RUP?}
\label{sec: Cuando debo utilizar RUP}
Se utiliza RUP desde el inicio de  un proyecto de software, y puede seguir utilizándolo en los ciclos de desarrollo subsiguientes tiempo después de que le proyecto inicial haya terminado. \par
La forma de utilizar RUP varía para ajustarse a sus necesidades. Existen unas pocas consideraciones que determinarán cuando y como utilizar partes diferentes de RUP.
\begin{itemize}
\item Ciclo vital del proyecto(numero de iteraciones, longitud de cada fase)
\item Propósitos empresariales, visión, ámbito y riesgo del proyecto
\item Tamaño del esfuerzo de desarrollo de software
\end{itemize}
En la siguiente figura \ref{fig:grafico} muestra a través de un gráfico de complejidad de técnica y gestión:

\begin{figure}[h]
	\centering
	\includegraphics[width=0.7\textwidth]{RUP.png}
	\caption{\label{fig:grafico}Gráfico complejidad de técnica y gestión.}
\end{figure}

\section{Fases de la metodología RUP}
Las fases indican el énfasis que se da en el proyecto en un instante dado. Para capturar la dimensión temporal de un proyecto, RUP divide el proyecto en cuatro fases diferentes(ver figura \ref{fig:fases}):

\begin{figure}[!h]
	\centering
	\includegraphics[width=0.8\textwidth]{fases.png}
	\caption{\label{fig:fases}Fases de RUP.}
\end{figure}

\begin{itemize}
\item Iniciación o Diseño: énfasis en el alcance del sistema
\item Preparación: énfasis en la arquitectura
\item Construcción: énfasis en el desarrollo
\item Transición: énfasis en la aplicación

\end{itemize}

\subsection{Fase de Inicio}


\chapter{Aplicación en Canchapp}

\chapter{Conclusión}


%\listoftables


\end{document}

